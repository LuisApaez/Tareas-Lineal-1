\documentclass{article}
\usepackage[utf8]{inputenc}
\usepackage[spanish,es-lcroman]{babel}
\usepackage{amsthm, amsmath, amssymb}

\usepackage{geometry}
\geometry{
    letterpaper,
    text={20cm,25cm},
    centering,
    top=1.5cm,
    bottom=1.5cm
}
\parindent  = 0mm
\parskip    = 4mm

\usepackage[usenames]{color}
\definecolor{azul}{RGB}{10,80,190}
\definecolor{negro}{RGB}{0,0,0}
\definecolor{rojo}{RGB}{200,40,40}
\definecolor{verde}{RGB}{0,120,50}

\newtheorem*{definition*}{Definición}
\renewcommand{\qedsymbol}{$\blacksquare$}
\usepackage{enumerate}

\begin{document}
    \title{Tarea 1. Espacios Vectoriales}
    \author{\bf Careaga Carrillo Juan Manuel, Luis Fernando Apáez Álvarez}
    \date{22 de febrero de 2019}
    \maketitle
    
    \begin{enumerate}
        % Ejercicio 1
        \item Sean $\mathbb{C}$ el campo de los números complejos, $\mathbb{R}$
        el campo de los números reales y $\mathbb{Q}$ el campo de los números
        racionales. Determine cuál de los siguientes es un espacio vectorial,
        en caso de ser espacio vectorial obtenga una base. Argumente su
        respuesta.
        \begin{enumerate}[i.]
            \item {
                $\mathbb{C}$ sobre $\mathbb{C}$
            }
            \item {
                $\mathbb{C}$ sobre $\mathbb{R}$
            }
            \item {
                $\mathbb{R}$ sobre $\mathbb{C}$
            }
            \item {
                $\mathbb{R}$ sobre $\mathbb{Q}$
            }
            \item {
                $\mathbb{Q}$ sobre $\mathbb{R}$
            }
            \item {
                $\mathbb{Q}$ sobre $\mathbb{Z}$ donde $\mathbb{Z}$ es el conjunto de números enteros.
            }
            \item {
                $S=\left\{a+b\sqrt{2}+c\sqrt{5}\big\vert a,b,c\in\mathbb{Q}\right\}$ sobre $\mathbb{Q}$.
            }
            \item {
                $S=\left\{a+b\sqrt{2}+c\sqrt{5}\big\vert a,b,c\in\mathbb{Q}\right\}$ sobre $\mathbb{R}$.
            }
            \item {
                $S=\left\{a+b\sqrt{2}+c\sqrt{5}\big\vert a,b,c\in\mathbb{Q}\right\}$ sobre $\mathbb{C}$.
            }
        \end{enumerate}

        % Ejercicio 2
        \item Sea $\mathbb{R}^+\cup\{0\}$ el conjunto de todos los números reales positivos. Defina las siguientes operaciones:
        $$x\boxplus y=xy \text{ para cualquiera } x,y\in\mathbb{R}^+$$ y
        $$a\boxdot x=x^a \text{ para } x\in\mathbb{R}^+ \text{ y } a\in\mathbb{R}$$
        \begin{enumerate}[i.]
            \item {
                Demuestre que $(\mathbb{R}^+\cup\{0\},\boxplus,\boxdot)$ es un espacio vectorial sobre $\mathbb{R}$.
            }
            \item {
                Obtenga su dimensión y una base
            }
            \item {
                Si se definiera $a\boxtimes x=a^x$ para $x\in\mathbb{R}^+$ y $a\in\mathbb{R}$ como la multiplicación por escalares, ¿$_\mathbb{R}(\mathbb{R}^+\cup\{0\})$?
            }
        \end{enumerate}

        % Ejercicio 3
        \item {
            Sea $V$ un $F$ espacio vectorial, $\{x,y,z\}\subseteq_{F}V$ un conjunto linealmente independiente. Determine el valor de $k\in F$ para el cual $$\{y-x,kz-y,x-z\}$$ es un conjunto linealmente independiente.
        }

        % Ejercicio 4
        \item Sea $\{x_1,x_2,\ldots,x_n\}$ una base para el espacio vectorial $_{F}V$, con $n\geq 2$.
        \begin{enumerate}[a.]
            \item {
                Demuestre que $$\left\{x_1,x_1+x_2,x_1+x_2+x_3,\ldots,\sum_{i=1}^{n}{x_i}\right\}$$ es una base para $_{F}V$
            }
            \item {
                El conjunto $$\{x_1+x_2,x_2+x_3,x_3+x_4,\ldots,x_{n-1}+x_n,x_n+x_1\}$$ ¿es base para $_{F}V$?
            }
            \item {
                Si $\{x_1+x_2,x_2+x_3,x_3+x_4,\ldots,x_{n-1}+x_n,x_n+x_1\}$ es base para $_{F}V$, ¿$\{x_1,x_2,\ldots,x_n\}$ es base para $_{F}V$?
            }
        \end{enumerate}

        % Ejercicio 5
        \item 
    \end{enumerate}
\end{document}
